
Pensando nas dificuldades que o elevado custo computacional associado a utilização da técnica de mutação de software trás, este trabalho se propõe a desenvolver uma técnica para otimizar a aplicação dos testes de mutação por meio do sistema VCM e verificar a efetividade da técnica em um ambiente de desenvolvimento ágil.


\section{Tecnologias associadas}

Para a realização deste trabalho foi escolhida como linguagem de programação o \textit{ECMAScript} \cite{ecma:spec}, conhecido como \textit{Javascript}, por ser a linguagem de programação presente nas principais implementações de navegadores \textit{web} \cite{ecma:mozilla} e ser a mais presente em projetos de \textit{software} de código aberto, segundo uma pesquisa feita na ultima década \cite{ieee:programming-langs}. Para o sistema de versionamento, foi escolhido o Git, por ser comumente utilizado no desenvolvimento de \textit{software} e em projetos de código aberto. Por fim, teve-se como base a biblioteca de mutação de \textit{software} \textit{Stryker Mutator}, por ser amplamente utilizada em \textit{ECMAScript}.

A \cref{tbl:cronograma} apresenta uma estimativa das etapas a serem realizadas durante
o desenvolvimento do trabalho.

\begin{table}[ht]
\centering
\begin{tabular}{c | c}
Etapa do projeto & Período Estimado \\ \hline
\hline
Estudo de implementações da técnica de teste de mutação  & Maio e Junho \\ \\ \hline
Estudo do código fonte do \textit{Stryker Mutator} \\ visando a criação de um \textit{plugin} & Junho \\ \\ \hline
Desenvolvimento de uma POC de \textit{plugin} \\ para o \textit{Stryker Mutator} & Julho \\ \\ \hline
Desenvolvimento de uma POC de \textit{plugin} \\ para o \textit{Stryker Mutator} & Agosto \\ \\ \hline
Medidas de desempenho da aplicação da técnica desenvolvida & Setembro \\ \\ \hline
Análise dos resultados, publicação do código fonte \\ e finalização do trabalho & Outubro \\ \hline
\end{tabular}
\caption{Cronograma de atividades}
\label{tbl:cronograma}
\end{table}

