No contexto do teste de software, uma das principais desvantagens da técnica de Teste de Mutação é o elevado tempo de processamento para a execução das análises em bases de código muito grandes. Este trabalho tem como objetivo desenvolver uma integração com o sistema de versionamento de código fonte Git, de maneira que seja possível realizar testes de mutação rápidos, melhorando a produtividade na utilização dessa técnica no desenvolvimento de software.

Para isso, foi desenvolvido um algoritmo para a aplicação das mutações apenas nas partes do código fonte que estão sendo atualizadas (diffs) para uma nova versão e serão coletadas métricas para avaliar a melhoria no tempo de processamento dos testes. O trabalho será consolidado com a criação de uma biblioteca ou framework para alguma linguagem de programação.

Por fim, este trabalho também se propõe a realizar um estudo sobre a aplicabilidade da técnica desenvolvida no contexto de engenharia de software e desenvolvimento ágil.
